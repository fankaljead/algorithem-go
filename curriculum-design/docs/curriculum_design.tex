\documentclass{ctexart}

\usepackage{lipsum}
\usepackage{hyperref}
\usepackage[margin=1in,left=1.5in,includefoot]{geometry}
\usepackage{listings}
\usepackage{xcolor}
\usepackage{amsmath}
\usepackage{geometry}
% \geometry{screen}
\lstset{
    numbers=left, 
    numberstyle= \tiny, 
    keywordstyle= \color{ blue!70  },
    commentstyle= \color{red!50!green!50!blue!50}, 
    frame=shadowbox, % 阴影效果
    rulesepcolor= \color{ red!20!green!20!blue!20  },
    escapeinside=`, % 英文分号中可写入中文
    xleftmargin=2em,xrightmargin=2em, aboveskip=1em,
    framexleftmargin=2em
} 
% Heading and Footer Stuff
\usepackage{fancyhdr}
\hypersetup{
    colorlinks=true,
    bookmarks=true,
    %bookmarksopen=false,
    % pdfpagemode=FullScreen,
    % pdfstartview=Fit,
    pdftitle={算法课程设计},
    pdfauthor={周翔辉}
}
\pagestyle{fancy}
\fancyhead{}
\fancyfoot{}
% For right foot page number
\fancyfoot[R]{第{\thepage}页}
\renewcommand{\headrulewidth}{0pt}
% 页面下方的线
\renewcommand{\footrulewidth}{0.5pt}

\newcommand\tab[1][1cm]{\hspace*{#1}}

\begin{document}

\begin{titlepage}
    \begin{center}
        \line(1,0){300} \\
        [0.25in]
        \huge{\bfseries 算法课程设计} \\
        [2mm]
        \line(1,0){200} \\
        [16cm]
    \end{center}

    \begin{flushright}
        \textsc{\large 周翔辉\\
            \# 11603080122 \\
            2018年12月\\
        }
    \end{flushright}
\end{titlepage}

\pagenumbering{roman}
\tableofcontents
\cleardoublepage

\newpage
\pagenumbering{arabic}

\setcounter{page}{1}
\section{基本的递归算法}
\subsection{二项式的计算}
完成二项式公式计算,即 $C_n^k = C_{n-1}^{k-1} + C_{n-1}^k$ 公式解释为了从n个不同元素
中抓取k个元素 ($C_n^k$),可以这样考虑,如果第一个元素一定在结果中,那么就需要从剩下的n-1个
元素中抓取k-1个元素 ($C_{n-1}^{k-1}$);如果第一个元素不在结果中,就需要从剩下的n1个元素中抓取k个元素
($C_{n-1}^k$)。要求分别采用以下方法计算,并进行三种方法所需时间的经验分析。
\end{document}
